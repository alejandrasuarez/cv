\documentclass[11pt,letterpaper,roman]{moderncv}

\moderncvtheme[blue]{classic} 
\usepackage{mycvconfig}

\firstname{\thefirstname{Alejandra Isabel \\}}
\familyname{\thelastname{Suárez Ramírez}}
\title{\normalfont \small Ingeniera en Gestión Informática}

\address{Federico Gana \# 3575}{Macul -- Santiago.}

\mobile{56.9.8.814.90.77 }                   
\phone{56.2.723.56.00}                     
\email{asuarez@nexussa.cl }                     
%\homepage{\vspace{-2cm} www.alumnos.inf.utfsm.cl/\textasciitilde agaspar}
%\photo[64pt]{../data/images/arnaldo}
%\quote{...}         

\begin{document}
\maketitle

%%%%%%%%%%%% ABSTRACT %%%%%%%%%%%%
%\section*{\textbf{Res\'umen}}
%\begin{scriptsize}
%\textsl
%Es madre de un peque\~no hace dos a\~no y 10 meses,
%Ingeniera en Gestión en Informática con 7 años de experiencia.  % WHO IS
%Se ha especializado en ingeniería y calidad de software como también en gestión de
%calidad y medio ambiente en organizaciones. % WHAT HAVE BEEN DOING
%En la actualidad trabaja en Tata Consulting Services (TCS)
%donde se desempeña como Audit and Compliance Manager. % ACTUALLY
%Particularmente interesada en trabajar en gestión de proceso de calidad. % WHAT WANT
%\end{scriptsize}
%\newline
	
	
%%%%%%%%%%%% LABORAL HISTORY %%%%%%%%%%%%
\section{Antecedentes Laborales}


\subsection{Antecedentes Generales en Ingeniería}
	\cventry{Enero 2015 a Mayo 2019}
	{Nexus S.A.} % Empresa
	{Ingeniero de Procesos} % cargo
	{Encargada de mantener el Sistema de Gestión de Calidad ISO9001 y de proyectos de mejora de servicios.
	}{}
	{Mac Iver 440, Santiago, Chile}
	\cventry{Octubre 2012 a Diciembre 2014}
	{Nexus S.A.} % Empresa
	{Jefe de Proyecto de Calidad} % cargo
	{Encargada de mantener el Sistema de Gestión de Calidad ISO9001 y de proyectos de mejora de servicios.
	}{}
	{Mac Iver 440, Santiago, Chile}

	\cventry{Octubre 2007 a Octubre 2012}
	{Tata Consultancy Services} % Empresa
	{Audit and Compliance Manager} % cargo
	{encargada del proceso de auditoría de la organización en Chile}
	{Planificación, gestión, realización y análisis de auditorías internas. Definición e implementación de mejora de procesos. 
	Auditor de procesos internos (esta'nderes utilizados: normas ISO 9001, ISO 20000, ISO 27000, CMMI, PCMM y otros.
	He conducido auditorías a projectos, grupos de soporte y distintas a'reas dentro de la organización. 
	Relatora de entrenamientos internos, tales como: UML, Calidad y testing, auditoría, inducción a nuevos empleados.
	Facilitadora de procesos para los distintos proyectos en curso en la organización.
	Apoyando en la generación de documentación, y las distintas actividades que estos deben realizar para poder cumplir con los procesos y procedimientos internos.
	}
	{Curicó 18, Santiago, Chile.}
	
	\cventry{Agosto 2006 a Octubre 2007}
	{Practia Consulting S.A} % Empresa
	{Consultor en Aseguramiento de Calidad de Software (QA)} % cargo
	{Trabajando en aseguramiento de la calidad como consultor}
	{específicamente con el cliente Indumotora en Proyecto  ``Repuesto '' y con el cliente
	VTR Global Com, como consultor encargado de revisión, diseño y ejecución de casos de pruebas
	}
	{Avenida Luis Thayer Ojeda 0191, Providencia, Santiago, Chile.}
	
	
	\cventry{Enero 2006 -- Agosto 2006}
	{SECI Consultores} % Empresa
	{Ingeniero de Testing} % cargo
	{Trabajando en aseguramiento de la calidad como consultor}
	{Forma parte del equipo de Aseguramiento de la Calidad para el proyecto ``SIGGE'' (Sistema de Garantías Estatales) del plan Auge del Ministerio de Salud, Chile.
	}
	{}
	
	
\section{Competencias Espec\'ificas}
\subsection{Competencias de Ingenier\'ia de Software}\
\cvline{Competencias en General}
	{Experencia en gestión de proyectos de software, y metodologías de desarrollo. 
	Elaboración de plan de proyecto, experiencia en elicitación en  de requerimientos 
	y elaboración de propuestas técnicas y económicas, utilización de UML para el análisis y diseño 
	de software. Breve experiencia en Arquitectura de Software.}
	

\cvline{Normas ISO 9001, 14001 y OSHAS 18001}
	{Nivel Medio, Sistema de Gestión de Calidad, Interpretación y Ana'lisis de la Norma Internacional ISO 9001:2008. ISO 14001:2004. Sistema de Gestión en Seguridad y Salud Ocupacional, Interpretación y Ana'lisis de la Norma Internacional OHSAS 18001:2007. Curso de Auditor Líder en Sistemas de Gestión Integrado Bajo ISO 9001:2008, ISO 14001:2004 y OHSAS 18001:2007.}

\section{Antecedentes Acad\'emicos}

%%%%%%%%%%%% STUDIES %%%%%%%%%%%%
% UNIVERSITY studies
\cventry{Diploma}
	{\textsc{Universidad de Chile}}
	{Diplomado en Gestión Integrada de Calidad Seguridad, Medio Ambiente}
	{2011}
	{}
	{Diagonal Paraguay 257, Santiago, Chile.}
	
\cventry{Postítulo}
	{\textsc{Universidad de Chile}}
	{Postítulo en Ingeniería y Calidad de Software}
	{2007}
	{}
	{Avenida Blanco Encalada 2120, Santiago, Chile.}
	
\cventry{Estudios Superiores}
	{\textsc{Universidad Nacional Andrés Bello (UNAB)}}
	{Ingeniería En Gestión Informática}
	{Licenciatura en Ingeniería y Título de Ingeniera en Gestión Informática}
	{Duración 5 años (egreso 2011)}
	{Avenida República 237, Santiago, Chile.}


% highschool studies
\cventry{Estudios Secundarios}
	{\textsc{Liceo Comercial Ñuñoa}}
	{egreso 2001}
	{}
	{}
	{Exequiel Fernandez 260, Ñuñoa, Santiago, Chile.}
% primary studies
\cventry{Estudios Primarios}
	{\textsc{Escuela Florentino Ameghino}}
	{}
	{}
	{}
	{C.M. de Alvear 1144, Buenos Aires, Argentina.}

\end{document}
