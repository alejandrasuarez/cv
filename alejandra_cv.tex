\documentclass[11pt,letterpaper,roman]{moderncv}

\moderncvtheme[blue]{classic} 
\usepackage{mycvconfig}

\firstname{\thefirstname{Alejandra Isabel \\}}
\familyname{\thelastname{Suarez Ramirez}}
\title{\normalfont \small Ingeniero Informático}

\address{Quilín \# 3352}{Macul -- Santiago.}

\mobile{56.9.8.814.90.77 }                   
\phone{56.2.293.22.64}                     
\email{alejandra.suarez@tcs.com }                     
%\homepage{\vspace{-2cm} www.alumnos.inf.utfsm.cl/\textasciitilde agaspar}
%\photo[64pt]{../data/images/arnaldo}
%\quote{...}         

\begin{document}
\maketitle

%%%%%%%%%%%% ABSTRACT %%%%%%%%%%%%
\section*{\textbf{Res\'umen}}
	\begin{scriptsize}
	\textsl
	Es madre de un peque\~no hace dos a\~no y 10 meses, Ingeniera en Gestión en Informática con 7 años de experiencia.  % WHO IS
	Se ha especializado en ingeniería y calidad de software como también en gestión de calidad y medio ambiente en organizaciones. % WHAT HAVE BEEN DOING
    En la actualidad trabaja en Tata Consulting Services (TCS) donde se desempeña como Audit and Compliance Manager. % ACTUALLY
   	Particularmente interesada en trabajar en gestión de proceso de calidad. % WHAT WANT
    \end{scriptsize}
    \newline

\section{Antecedentes Acad\'emicos}


%%%%%%%%%%%% STUDIES %%%%%%%%%%%%
% UNIVERSITY studies
\cventry{Diploma}
	{\textsc{Universidad de Chile}}
	{Diplomado en Gestión Integrada de Calidad Seguridad, Medio Ambiente}
	{2011}
	{}
	{Diagonal Paraguay 257, Santiago, Chile.}
	
\cventry{Postítulo}
	{\textsc{Universidad de Chile}}
	{Postítulo en Ingeniería y Calidad de Software}
	{2007}
	{}
	{Avenida Blanco Encalada 2120, Santiago, Chile.}
	
\cventry{Estudios Superiores}
	{\textsc{Universidad Nacional Andrés Bello (UNAB)}}
	{Ingenier\'ia En Gestión Inform\'atica}
	{Licenciatura en Ingeniería y Título de Ingeniera en Gestión Informática	}
	{Duración 5 años (egreso 2011)}
	{Avenida República 237, Santiago, Chile.}


% highschool studies
\cventry{Estudios Secundarios}
	{\textsc{Liceo Comercial Ñuñoa}}
	{egreso 2001}
	{}
	{}
	{Exequiel Fernandez 260, Ñuñoa, Santiago, Chile.}
% primary studies
\cventry{Estudios Primarios}
	{\textsc{Escuela Florentino Ameghino}}
	{}
	{}
	{}
	{C.M. de Alvear 1144, Buenos Aires, Argentina.}

%%%%%%%%%%%% OTHER STUDIES %%%%%%%%%%%%
\subsection{Cursos Internos y Contratados}
Cursos Internos dictados por profesionales internos o contratados para este propósito:

\cventry{Julio 2009}
	{Calidad y Testing}
	{Relator Viviana Laureyro}
	{}
	{Practia Consulting S.A.}
	{Santiago, Chile}



\cventry{Julio 2009}
	{DAR - Six Thinking Hats Training}
	{}
	{}
	{Practia Consulting S.A.}
	{Santiago, Chile}

\cventry{Julio 2009}
	{IPRE and Prediction Models}
	{}
	{}
	{Practia Consulting S.A.}
	{Santiago, Chile}
	
\cventry{Julio 2009}
	{Control estadístico de Procesos}
	{}
	{}
	{Practia Consulting S.A.}
	{Santiago, Chile}
	
	
\cventry{Julio 2009}
	{Técnicas de Control Estadístico}
	{}
	{}
	{Tata}
	{Santiago, Chile}
	
\cventry{Julio 2009}
	{Gestión de Riesgos}
	{Curso dictado por Viviana Laureyro}
	{}
	{}
	{Santiago, Chile}

\cventry{Julio 2009}
	{Defect Prevention \& Causal Analysis Techniques Course}
	{Curso dictado por Viviana Laureyro}
	{}
	{Practia Consulting S.A.}
	{Santiago, Chile}
	

\cventry{Julio 2009}
	{PDCA Course}
	{}
	{}
	{}
	{}
	
	
\cventry{Julio 2009}
	{ISO 27001 Course}
	{}
	{}
	{}
	{}
	
\cventry{Julio 2009}
	{ISO 20000 Course}
	{}
	{}
	{}
	{}
\cventry{Julio 2009}
	{SAS 70 compliance Awareness Course}
	{}
	{}
	{}
	{}
\cventry{Julio 2009}
	{Oriented Analysis and Design with UML Course}
	{}
	{}
	{}
	{}
\cventry{Julio 2009}
	{Introduction to Auditing Course}
	{}
	{}
	{}
	{}
	
\cventry{Julio 2009}
	{Internal Auditors Training Course}
	{}
	{}
	{}
	{}
	
\cventry{Julio 2009}
	{Project metrics \& Improvement Course}
	{}
	{}
	{}
	{}
\cventry{Julio 2009}
	{CMMi.}
	{}
	{}
	{}
	{}
	
	
%%%%%%%%%%%% LABORAL HISTORY %%%%%%%%%%%%
\section{Antecedentes Laborales}


\subsection{Antecedentes Generales en Ingeniería}
	\cventry{Octubre 2007 a la fecha}
	{Tata Consultancy Services} % Empresa
	{Audit and Compliance Manager} % cargo
	{encargada del proceso de auditoría de la organización en Chile}
	{Planificación, gestión, realización y análisis de auditorías internas. Definición e implementación de mejora de procesos. 
	Auditor de procesos internos (estánderes utilizados: normas ISO 9001, ISO 20000, ISO 27000, CMMI, PCMM y otros.
	He conducido auditorías a projectos, grupos de soporte y distintas áreas dentro de la organización. 
	Relatora de entrenamientos internos, tales como: UML, Calidad y testing, auditoría, inducción a nuevos empleados.
	Facilitadora de procesos para los distintos proyectos en curso en la organización.
	Apoyando en la generación de documentación, y las distintas actividades que estos deben realizar para poder cumplir con los procesos y procedimientos internos.
	}
	{Curicó 18, Santiago, Chile.}
	
	\cventry{Agosto 2006 a Octubre 2007}
	{Practia Consulting S.A} % Empresa
	{Consultor en Aseguramiento de Calidad de Software (QA)} % cargo
	{Trabajando en aseguramiento de la calidad como consultor}
	{específicamente con el cliente Indumotora en Proyecto  ``Repuesto '' y con el cliente
	VTR Global Com, como consultor encargado de revisión, diseño y ejecución de casos de pruebas
	}
	{Avenida Luis Thayer Ojeda 0191, Providencia, Santiago, Chile.}
	
	
	\cventry{Enero 2006 -- Agosto 2006}
	{SECI Consultores} % Empresa
	{Ingeniero de Testing} % cargo
	{Trabajando en aseguramiento de la calidad como consultor}
	{Forma parte del equipo de Aseguramiento de la Calidad para el proyecto ``SIGGE'' (Sistema de Garantías Estatales) del plan Auge del Ministerio de Salud, Chile.
	}
	{}
	

\subsection{Otros}

	\cventry{Enero 2006 -- Agosto 2006}
	{Universidad Academia de Humanismo Cristiano.} % Empresa
	{Ayudante en Contabilidad} % cargo
	{Ayudante en Departamento de Contabilidad durante el proceso de postulación y inscripción de alumnos.}
	{}
	{Condell 343 Providencia Santiago de Chile}
	%Huérfanos 2186 y 1869, Santiago, Chile
	
	
\section{Competencias Espec\'ificas}
\subsection{Competencias de Ingenier\'ia de Software}\
\cvline{Competencias en General}
	{Experencia en gestión de proyectos de software, y metodologías de desarrollo. 
	Elaboración de plan de proyecto, experiencia en elicitación en  de requerimientos 
	y elaboración de propuestas técnicas y económicas, utilización de UML para el análisis y diseño 
	de software. Breve experiencia en Arquitectura de Software.}
	

\cvline{Normas ISO 9001, 14001 y OSHAS 18001}
	{Nivel Medio, Sistema de Gestión de Calidad, Interpretación y Análisis de la Norma Internacional ISO 9001:2008. ISO 14001:2004. Sistema de Gestión en Seguridad y Salud Ocupacional, Interpretación y Análisis de la Norma Internacional OHSAS 18001:2007. Curso de Auditor Líder en Sistemas de Gestión Integrado Bajo ISO 9001:2008, ISO 14001:2004 y OHSAS 18001:2007.}

\subsection{Competencias Computacionales}
\cvline{CMMI}
	{Nivel Medio}

\cvline{Six Sigma}
	{Nivel Usuario}
\cvline{Herramientas de Configuración,}
	{(Harvest, CVS, ClearQuest}

Project, Nivel Medio 
Enterprise Architect, Nivel Medio 
BPMN, Nivel Usuario
ISO 20000, Nivel Usuario 
ISO 27001, Nivel Usuario


\url{https://github.com/alejandrasuarez/cv.git}

\end{document}
